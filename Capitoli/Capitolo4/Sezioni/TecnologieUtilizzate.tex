\section{Tecnologie utilizzate}
Al fine di testare l'algoritmo progettato, abbiamo sviluppato una applicazione scritta in Java.
La scelta nell'utilizzare Java, come detto negli studi preliminari, è nata dalla necessità di ottenere maggiori prestazioni, dalla potenza del linguaggio e dal gran numero di librerie e framework utilizzabili al nostro scopo.
Nella seguente sezione analizzeremo quelle che sono le librerie utilizzate per la realizzazione del seguente progetto.
\subsection{AKKA}
Akka è un insieme di strumenti per la realizzazione di applicazioni Java o Scala altamente concorrenti e distribuite. 
La scelta di questo framework è stata fatta in seguito alla necessità di rendere l'algoritmo quanto più efficiente possibile. La soluzione migliore si è rivelata essere quella del multithreading. A questo punto si è pensato che una impostazione \emph{low level} dei thread fosse poco sicura, quindi ci siamo orientati verso l'utilizzo di un container che gestisse le operazioni fondamentali di coordinazione.
Oltre alla coordinazione del multithreading \textbf{AKKA} ci offre la possibilità di creare una applicazione distribuita.
Questo argomento è stato trattato in fase di studio preliminare e temporaneamente accantonato, ciò non toglie che l'utilizzo di questo framework renda l'applicazione predisposta alla distribuzione.
AKKA ci offre un \emph{ambiente sicuro} su cui eseguire i nostri thread chiamato \textbf{Actor system}.
L'Actor system è un container che prevede la gestione dei vari thread, chiamati \textbf{Actor}, garantendo servizi come la scalabilità della nostra applicazione, gestione della concorrenza e massime prestazioni.
Per questa tecnologia si può fare riferimento alla documentazione ufficiale: \url{https://akka.io}
\subsection{FastUtil}
In fase di implementazione si è reso necessario un boost nelle prestazioni ed un uso efficiente della memoria da parte delle principali strutture dati utilizzate da Java.
Per garantire questi servizi ci viene incontro una libreria chiamata \textbf{fastutil}.
Questa libreria re-implementa le principali strutture di java garantendoci i servizi succitati.
Le strutture sfruttate nella nostra applicazione sono:
\begin{itemize}
	\item \textbf{\emph{ObjectArrayList}}: Re-implementazione della classe ArrayList di Java.
	\item \textbf{\emph{Object2ObjectHashMap}}: Re-implementazione della classe HashMap di Java.
\end{itemize}
\subsection{Joinery Dataframe}
Il DataFrame di joinery è una struttura dati simile a dataframe presente nella libreria \emph{pandas} del linguaggio Python.
Essa ci permette di memorizzare velocemente un \emph{dataset} ed effettuare operazioni su di esso come se fosse una semplice tabella.
La scelta di questa libreria è stata fatta in seguito alla valutazione dei vantaggi che offrivano i vari metodi implementati da Joinery.
Si sono mostrati essere molto vantaggiosi i metodi che garantiscono l'import di un dataset da file csv e la gestione diretta ai dati in esso contenuti.
Si può fare riferimento a tale libreria sulla documentazione ufficiale: \url{https://cardillo.github.io/joinery/}.
