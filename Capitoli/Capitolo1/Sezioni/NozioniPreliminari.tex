\section{Nozioni Preliminari}
Prima di esporre le RFD è necessario introdurre alcuni concetti preliminari.

\paragraph{Schema di relazione}
Uno schema di relazione è costituito da un simbolo $R$, detto nome della relazione, e da un insieme di attributi $X = \{A_1,A_2,...,A_n\}$, di solito indicato
con $R(X)$. A ciascun attributo $A \in X$ e associato un dominio $dom(A)$.
Uno schema di base di dati è un insieme di schemi di relazione con nomi
diversi:
\\~\\
\centerline{$R = \{ R_1(X_1),R_2(X_2),\ldots,R_n(X_n)\}$.}
\\~\\
Una relazione su uno schema $R(X)$ è un insieme $r$ di tuple su $X$. Per
ogni istanza $r \in R(X)$, per ogni tupla $t \in r$ e per ogni attributo $A \in X$,
$t[A]$ rappresenta la proiezione di $A$ su $t$. In modo analogo, dato un insieme
di attributi $Y \subseteq X$, $t[Y]$ rappresenta la proiezione di $Y$ su $t$.\cite{libroCeri}


\begin{table}[H]
    \centering
    \begin{tabular}{ | l | l | l | l |}
        \hline
        Matricola & Cognome & Nome & Data di nascita\\
        \hline
        123456 & Rossi & Maria & 25/11/1991 \\ 
        654321 & Neri & Anna & 23/04/1992 \\ 
        456321 & Verdi & Fabio & 12/02/1992 \\
        \hline
    \end{tabular}
    \caption{Esempio di schema di relazione}
    \label{tab:table example}
\end{table}
