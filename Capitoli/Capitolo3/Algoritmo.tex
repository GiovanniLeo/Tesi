In questo capitolo saranno mostrati i passi da effettuare per ottenere, partendo da un dataset rappresentante una relazione, una lista di dipendenze funzionali rilassate.
La sequenza di passi che l'algoritmo affronterà sono:
\begin{itemize}
	\item \textbf{\emph{Feasibility}}
	\item \textbf{\emph{Minimality}}
	\item \textbf{\emph{Generation}}
\end{itemize}
Per fare in modo che la prima fase(\emph{Feasibility}) abbia inizio, ci dobbiamo creare la matrice delle distanze, che, insieme ad alcune informazioni aggiuntive verranno date in input alla suddetta fase.
In seguito si passerà alla fase di \textit{Minimality} la quale fornirà l'input per l'ultima fase ovvero, \textit{Generation}. In questo capitolo verrà descritta nello specifico la fase di \textit{Generation} che è oggetto di questo lavoro di tesi.
\import{Capitoli/Capitolo3/Sezioni/}{MatriceDelleDistanze}
\import{Capitoli/Capitolo3/Sezioni/}{RFDGeneration}