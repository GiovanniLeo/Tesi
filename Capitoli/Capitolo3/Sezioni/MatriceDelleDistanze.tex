\section{Matrice delle distanze}
Il primo passo consiste nel calcolo delle distanze tra ogni coppia di tuple del dataset. Questo passaggio viene fatto utilizzando diverse funzioni di distanza, a seconda del tipo di RFD che si vuole ricavare.
\subsection{Funzione di distanza}
Date due tuple $t_i$ e $t_j$ tali che $t_i, t_j \in \mathbf{r}$, risulta necessario definire un metodo per capire quanto queste siano distanti tra di loro. Definiamo quindi una funzione 
$$ \mathbf{d}: \mathbf{r} \rightarrow \mathbb{R}^{k+1} $$
tale che, date due tuple $t_i , t_j \in \mathbf{r}$, 
$$ \mathbf{d}(t_i, t_j) = [(d_1, d_2,..., d_{k+1}) \mid d_p = d(t_i[p], t_j[p])  ], $$
dove $t_i[p]$ rappresenta il p-esimo elemento della tupla $t_i$ e $d(x,y)$ è una funzione di distanza tra gli elementi $x$ e $y$ che cambia in base al tipo di elemento. \\
Una funzione $\mathbf{d}(t_i, t_j)$ può quindi essere ibrida, ossia composta da diverse distanze su attributi di tipi diversi: per confrontare tuple composte da valori misti (e.g. numerici, stringhe e date) è necessario utilizzare diverse distanze. Quindi, una generica funzione distanza $\mathbf{d}$ operante su due tuple, è composta al suo interno da altre funzioni di distanza $d_{t}$, anche diverse tra di loro, operanti su singoli attributi.\\
Durante la progettazione del nostro algoritmo abbiamo considerato le seguenti funzioni distanza, in base ai diversi tipi di elemento che potevano far parte di una tupla:
\paragraph{Distanza euclidea mono-dimensionale} tale funzione viene utilizzata quando $x$ e $y$ sono entrambi tipi numerici:
\begin{equation}
	d_{eucl}(x,y) = \sqrt[]{(x-y)^{2}} = |x - y|.
\end{equation}
\paragraph{Distanza di Levenshtein} è una metrica per misurare la differenza fra due stringhe (o sequenze in generale). Tale distanza indica il minimo numero di caratteri singoli che è necessario modificare (e.g. inserimenti, cancellazioni, sostituzioni) per far sì che le due stringhe siano uguali. È quindi definita come:
\begin{equation}
	d_{lev}(x,y) = lev_{x,y}(|x|,|y|)
\end{equation}
dove 
$$lev_{a,b}(i,j) = 
\begin{cases}
	\max(i,j) & \text{se}\ min(i,j) = 0 \\
    \min{ 
    	\begin{cases}
    		lev_{a,b}(i-1,j) + 1 \\
           	lev_{a,b}(i,j-1) + 1 \\
            lev_{a,b}(i-1,j-1) + 1_{(a_i \neq b_j)}
    	\end{cases}} & \text{altrimenti}
\end{cases}
$$
dove $1_{(a_i \neq b_j)}$ è una \textit{funzione indicatrice}, uguale a 0 se $a_i = b_j$ e uguale a 1 altrimenti, e $lev_{a,b}(i,j)$ è la distanza di Levenshtein  tra i primi $i$ caratteri di $a$ e i primi $j$ caratteri di $b$;
\paragraph{Distanza tra date} quando gli attributi da confrontare sono di tipo data (a prescindere dal formato di questa), viene utilzzata una semplice funzione di distanza che permette di calcolare i giorni trascorsi dalla data meno recente a quella più recente. 
\subsection{Calcolo della matrice delle distanze}
Dal confronto tra tutte le coppie di tuple della relazione ricaviamo una tabella delle distanze $DT$ tale che $|DT| = \binom{n}{2} = \frac{n(n-1)}{2}$.  \\ 
La tabella delle distanze (supponendo senza perdere di generalità di utilizzare l'attributo $X_{k+1}$ come attributo RHS) ricavata è quindi:
\begin{center}
	\begin{tabular}{l|l|llll}
		{} &  $RHS$ & $X_1$ &   $X_2$ &  $\ldots$ &   $X_k$ \\
		1 & $d^{RHS}_1$ & $d_{11}$ &  $d_{12}$ &  $\ldots$ &  $d_{1k}$ \\
		2 & $d^{RHS}_2$ & $d_{21}$ &  $d_{22}$ &  $\ldots$ &  $d_{2k}$ \\
		\vdots & \vdots & \vdots &  \vdots &  $\ddots$ &  \vdots \\
		m & $d^{RHS}_m$ &  $d_{m1}$ &  $d_{m2}$ &  \ldots &  $d_{mk}$ \\
	\end{tabular} \\
\end{center}
dove $m = \frac{n(n-1)}{2}$ e ogni riga rappresenta il vettore distanza ottenuto confrontando due diverse tuple facenti parte della relazione, ad esempio, date due tuple $t_r = (x_1, x_2, \ldots, x_{k+1})$ e $t_l = (y_1, y_2, \ldots, y_{k+1})$ tali che $t_r \in \mathbf{r}$ e $t_l \in \mathbf{r}$, calcoliamo la distanza tra queste due tuple e poniamo il vettore 
$$d(t_r,t_l) = [d(x_1, y_1), d(x_2, y_2), \ldots, d(x_{k+1}, y_{k+1})]$$
in $DT[i]$. Il generico $d_{ij}$ rappresenta la distanza sull'attributo $X_i$ calcolato su una coppia di tuple. \\
La colonna 
$$DT[\cdot]['RHS'] = {d_1^{RHS} \\ d_2^{RHS} \\ \vdots \\ d_m^{RHS}}$$
rappresenta la colonna delle distanze tra tutte le $n$ tuple rispetto all'attributo RHS. \\ \\
Dopo aver calcolato la tabella $DT$, occorre ordinarla utilizzando come pivot la colonna $DT[\cdot]['RHS']$, ottenendo il seguente data frame: \\
\begin{center}
	\begin{tabular}{l|l|llll}
		{} &  $RHS$ & $X_1$ &   $X_2$ &  $\ldots$ &   $X_k$ \\
		1 & $d^{RHS}_i$ & $d_{i1}$ &  $d_{i2}$ &  $\ldots$ &  $d_{ik}$ \\
		2 & $d^{RHS}_j$ & $d_{j1}$ &  $d_{j2}$ &  $\ldots$ &  $d_{jk}$ \\
		\vdots & \vdots & \vdots &  \vdots &  $\ddots$ &  \vdots \\
		m & $d^{RHS}_r$ &  $d_{r1}$ &  $d_{r2}$ &  \ldots &  $d_{rk}$ \\
	\end{tabular} \\
\end{center}
dove $d^{RHS}_i \leq d^{RHS}_j \leq \ldots \leq d^{RHS}_r$. \\
A questo punto occorre dividere e raggruppare tutte le righe del data frame: le righe aventi lo stesso valore(\textit{ClusterID}) sulla colonna $DT[\cdot]['RHS']$ saranno raggruppate in sezioni di data frame disgiunte. 
Questo data frame finale può finalmente essere utilizzato dalle successive fasi dell'algoritmo.