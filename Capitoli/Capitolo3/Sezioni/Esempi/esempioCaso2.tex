Consideriamo a questo punto un esempio:


\begin{table}[H]
	\centering
	\begin{tabular}{l l l l l}
		TID & IDCheck & PresDate & ExeDate & RealDate \\
		\hline
		1 & 2 & 2/7/17 & 15/7/17 & 18/7/17\\
		2 & 3 & 3/7/17 & 16/7/17 & 20/7/17\\
		3 & 1 & 2/7/17 & 16/7/17 & 19/7/17\\
		4 & 1 & 5/7/17 & 19/7/17 & 19/7/17\\
		5 & 2 & 8/7/17 & 19/7/17 & 23/7/17\\
	
	\end{tabular}
	\caption{Dataset di esempio.}
	\label{tab:Dataset_di_esempio}
\end{table}

\begin{table}[H]
	\centering
	\begin{tabu}{l l | l l l}
		PairID & ExeDate & IDCheck & PresDate & RealDate \\
		\hline
		2,3 & 0 & 0 & 1 & 1\\
		 \rowfont{\color{gray}}
		4,5 & 0 & 1 & 3 & 4 \\
		\hline
		1,2 & 1 & 0 & 1 & 2\\
		1,3 & 1 & 1 & 0 & 1\\
		\hline
		2,4 & 3 & 0 & 2 & 1\\
		\rowfont{\color{gray}}
		2,5 & 3 & 1 & 5 & 3 \\
		2,3 & 3 & 1 & 3 & 0\\
		\rowfont{\color{gray}}
		4,5 & 3 & 0 & 6 & 4 \\
		\hline
		1,4 & 4 & 0 & 3 & 1\\
		\rowfont{\color{gray}}
		1,5 & 4 & 1 & 6 & 3 \\
		
		
	\end{tabu}
	\caption{Distance Matrix di esempio.}
	\label{tab:Distance_Matrix_di_esempio}
\end{table}
Sia \textbf{$X$}=PresDate,RelDate \textbf{$A$}=ExeDate e \textbf{$\epsilon-step$}=1. Allora
\begin{center}
      $PresDate_{(\leq m_{1} - 1)} RelDate_{(\leq p_{2} - 1)}\rightarrow ExeDate_{(\leq next(k))}$ \\
    $PresDate_{(\leq p_{1} - 1)} RelDate_{(\leq m_{2} - 1)}\rightarrow ExeDate_{(\leq next(k))}$
\end{center}
Sappiamo che la coppia minima la otteniamo dalla fase di \textit{Minimality} ed ad essa è associata un determinato pattern.
Guardando nella tabella sottostante possiamo notare che la coppia $\{3,1\}$ evidenziata non è una coppia minima ma è una coppia che non domina e quindi non avendo altre coppie che non dominano non posso generare RFD. 
\begin{table}[H]
	\centering
	\begin{tabu}{l l | l!{\color{red}\vrule} l l!{\color{red}\vrule}}
		PairID & ExeDate & IDCheck & PresDate & RealDate \\
		\hline
		2,3 & 0 & 0 & 1 & 1\\
		\rowfont{\color{gray}}
		4,5 & 0 & 1 & 3 & 4 \\
		\hline
		1,2 & 1 & 0 & 1 & 2\\
		1,3 & 1 & 1 & 0 & 1\\
		\hline
		2,4 & 3 & 0 & 2 & 1\\
		\rowfont{\color{gray}}
		2,5 & 3 & 1 & 5 & 3 \\
		2,3 & 3 & 1 & 3 & 0\\
		\rowfont{\color{gray}}
		4,5 & 3 & 0 & 6 & 4 \\
		\hline
		1,4 & 4 & 0 & \cellcolor{red!25}{3} & \cellcolor{red!25}{1}\\
		\rowfont{\color{gray}}
		1,5 & 4 & 1 & 6 & 3 \\	
	\end{tabu}
\end{table}
Passando al cluster successivo possiamo notare che la coppia $\{3,1\}$ domina e quindi non viene considerata.
Devo invece considerare $\{3,0\}$ e $\{2,1\}$ le quali non si dominano l'una con l'altra.
Se a questo punto considero la coppia $\{3,0\}$ ovvero quella evidenziata nella tabella sottostante
\begin{table}[H]
	\centering
	\begin{tabu}{l l | l!{\color{red}\vrule} l l!{\color{red}\vrule}}
		PairID & ExeDate & IDCheck & PresDate & RealDate \\
		\hline
		2,3 & 0 & 0 & 1 & 1\\
		\rowfont{\color{gray}}
		4,5 & 0 & 1 & 3 & 4 \\
		\hline
		1,2 & 1 & 0 & 1 & 2\\
		1,3 & 1 & 1 & 0 & 1\\
		\hline
		2,4 & 3 & 0 & 2 & 1\\
		\rowfont{\color{gray}}
		2,5 & 3 & 1 & 5 & 3 \\
		2,3 & 3 & 1 & \cellcolor{red!25}{3} & \cellcolor{red!25}{0}\\
		\rowfont{\color{gray}}
		4,5 & 3 & 0 & 6 & 4 \\
		\hline
		1,4 & 4 & 0 & 3 & 1\\
		\rowfont{\color{gray}}
		1,5 & 4 & 1 & 6 & 3 \\	
	\end{tabu}
\end{table}
Allora secondo le regole viste in precedenza si ha che  che $m_{1} - 1 = 2$ e $p_{1} - 1 = - 1$(essendo $p_{1} = 0$) non potremmo generare nessuna RFD poiché $p_{1} - 1 < 0$. A questo punto devo andare a prendere tutti quei valori in cui $m_{1}$ è strettamente maggiore, si da il caso che in questo cluster 3 sia maggiore di 2. In questo caso $p_{2}=1$ il quale deve essere maggiore o uguale di $p_{1}$(in questo caso lo è) e $m_{1} - 1$ deve essere maggiore o uguale di $p_{2}$

\begin{table}[H]
	\centering
	\begin{tabu}{l l | l!{\color{red}\vrule} l l!{\color{red}\vrule}}
		PairID & ExeDate & IDCheck & PresDate & RealDate \\
		\hline
		2,3 & 0 & 0 & 1 & 1\\
		\rowfont{\color{gray}}
		4,5 & 0 & 1 & 3 & 4 \\
		\hline
		1,2 & 1 & 0 & 1 & 2\\
		1,3 & 1 & 1 & 0 & 1\\
		\hline
		2,4 & 3 & 0 & \cellcolor{red!25}{2} & \cellcolor{red!25}{1}\\
		\rowfont{\color{gray}}
		2,5 & 3 & 1 & 5 & 3 \\
		2,3 & 3 & 1 & 3 & 0\\
		\rowfont{\color{gray}}
		4,5 & 3 & 0 & 6 & 4 \\
		\hline
		1,4 & 4 & 0 & 3 & 1\\
		\rowfont{\color{gray}}
		1,5 & 4 & 1 & 6 & 3 \\	
	\end{tabu}
\end{table}
Allora a questo punto si ha che 
\begin{equation*}
	PresDate_{(\leq 2)} RelDate_{(\leq 0)}\rightarrow ExeDate_{(\leq 1)}
\end{equation*}
la quale è una RFD valida.
%!{\color{red}\vrule}