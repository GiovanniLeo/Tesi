
\section{Feasibility e Minimality}
La fase di \textit{Feasibility} sottopone i pattern ad un test e restituisce l'insieme dei pattern che li hanno superati. Ce ne sono tanti quanti sono gli RHS. Invece la fase si \textit{Minimality} restituisce in output un certo numero di sottopattern minimi.
\section{RFD Generation}
\subsection{Nozioni Preliminari}
Dalla fase di \textit{Minimality} si otterranno un certo numero di sottopattern minimi(ovvero minimali\footnote{Non è possibile eliminare nessun attributo da tale pattern poiché eliminandolo la distanza di un attributo diventa dominante.} e ammissibili\footnote{Tali pattern non dominano altri pattern.}) i quali possono essere denominati $S_{k}$. Essendo che ogni RHS contiene un certo numero di $C_{i}$\footnote{Gli insiemi $C_{i}$ sono l'output della fase di  \textit{Feasibility}.} (con $i=1,\dots,m$)  dove ogni $C_{i}$ contiene un certo numero di pattern $P_{j}$ (con $j = 1,\dots,h$) e per ognuno di questi pattern si otterrà un $S_{k}$
\subsection{Identificazione di soglie ottime}
L'idea generale è quella di trovare i più grandi pattern di thresholds tali che questi non dominano i pattern che non dominano. Quindi dato un $S_{k}$ ammissibile dobbiamo generare le soglie ottime.\\
L'idea di base è una volta che ho trovato un minimo per un cluster, devo generare le RFD per il cluster successivo, poiché una volta raggiunto il minimo viene trovata una violazione alla dipendenza candidata questo implica che la soglia valida per il cluster successivo non deve raggiunge il minimo. La regola generale è che ogni volta che trova una coppia che è simile sull' LHS(in questa sezione lo chiameremo $X$) questa deve essere simile secondo quella soglia sull'RHS. Possiamo distinguere 3 casi.
\begin{itemize}
	\item 
		\textbf{Caso base:} Consideriamo un unico attributo candidato per $X$($|X|= 1$).\\
		Dato un \textit{clusterID} $k>0$ dell'attributo $A$, se esiste trova il minimo valore $m$ tale che 
		\begin{equation}
			m=min(k)<min(prev(k)\footnote{Indica il \textit{clusterID} precedente}) 
		\end{equation}
		$(m-\epsilon)$ rappresenta la migliore soglia per il cluster $next(k)$\footnote{Indica il \textit{clusterID} successivo}, in questo caso $\epsilon = 1$. Se $m > 0$ (non avrebbe senso imporre che la soglia sia minore o uguale ad un numero negativo) allora genero 
		\begin{equation}
		 X_{(\leq m-\epsilon)} \rightarrow A_{(\leq next(k))}
		\end{equation}
		nel caso in cui $m=min(prev(k))$ devo aggiornare 
		\begin{center}
		$X_{(\leq m-\epsilon)} \rightarrow A_{(\leq k)}$
		in $ X_{(\leq m-\epsilon)} \rightarrow A_{(\leq next(k))}$
		\end{center}
		in altre parole quando ho due o più RFD sullo stesso LHS devo scegliere quella con RHS minore.
	\item 
		\textbf{Caso con $|X|=2$:} Dato un \textit{clusterID} $k>0$ dell'attributo $A$, se esiste trova le coppie $(m_{1},m_{2})$ non dominanti rispetto all'insieme di tuple \\$S=nonDominating(k)$, quindi
		$(m_{1}-\epsilon,\alpha_{2})$ rappresenta una delle migliori soglie per il cluster $next(k)$.\\
		Se $m_{1}>0$ genera
		\begin{equation}
		(X_{1})_{(\leq m_{1}-\epsilon)} (X_{2})_{(\leq \alpha_{2})}\rightarrow A_{(\leq next(k))}
		\end{equation}
		invece,
		$(\alpha_{1},m_{2}-\epsilon)$ rappresenta una delle migliori soglie per il cluster $next(k)$.\\
		Se $m_{2}>0$ genera
		\begin{equation}
		(X_{1})_{(\leq \alpha_{2})} (X_{2})_{(\leq m_{2}-\epsilon)}\rightarrow A_{(\leq next(k))}
		\end{equation}
		il tutto è possibile se $\alpha_{j}$(in questo caso $j=1,2$) esiste. A questo punto devo determinare $\alpha_{j}$ per un $X_{j}$ di cui viene fatto il dicovery secondo $m_{i}$ di $X_{i}$ ($j=1,2$, $i=1,2$ in particolare quando $j=1$, $i=2$ e viceversa)per un dato pattern di tuple $t$.\\
		Consideriamo i pattern di tuple non dominanti $S=t_{1},...,t_{n}$ con $t_{l}\neq t$ per $l=1,...,n$.\\
		Se esiste, trova il minimo valore $p_{j}$ del pattern di tuple $t_{l}$ tale che
		\begin{enumerate}
			\item  $m_{j}<p_{j}$, e
			\item $m_{i-\epsilon} \geq p_{i}$ con $p_{i}=t_{l}[X_{i}]$
		\end{enumerate}  
	e genera $\alpha_{j}= p_{j} -\epsilon$.\\ In altre parole il minimo valore fra questi è dominato da $m_{i-\epsilon}$
		                                                                                                                                                                       
\end{itemize}

\subsection{Generazione di RFD}