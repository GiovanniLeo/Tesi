\section{Feasibility e Minimality}
La fase di \textit{Feasibility} sottopone i pattern ad un test e restituisce l'insieme dei pattern che li hanno superati. Ce ne sono tanti quanti sono gli RHS. Invece la fase si \textit{Minimality} restituisce in output un certo numero di sottopattern minimi.
\section{Nozioni Preliminari}
Dalla fase di \textit{Minimality} si otterranno un certo numero di sottopattern minimi i quali possono essere denominati $S_{k}$. Essendo che ogni RHS contiene un certo numero di $C_{i}$\footnote{Gli insiemi $C_{i}$ sono l'output della fase di  \textit{Feasibility}.} (con $i=1,\dots,m$)  dove ogni $C_{i}$ contiene un certo numero di pattern $P_{j}$ (con $j = 1,\dots,h$) e per ognuno di questi pattern si otterrà un $S_{k}$

\section{Identificazione di soglie ottime}
\section{Generazione di RFD}