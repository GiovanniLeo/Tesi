\section{Riflessioni}
Dopo aver completato la descrizione del progetto sviluppato è necessario fare qualche osservazione.
L'idea iniziale era di apportare alcune migliorie ad un algoritmo sviluppato in un precedente progetto di intelligenza artificiale\cite{tesinaIA}. Dopo alcune attente riflessioni, si è deciso di riprogettare l'algoritmo in un ambiente differente che è quello di \emph{JAVA}.
Il sistema, seppur necessitante di diverse migliorie, è stato sviluppato in modo da essere facilmente manutenibile. I diversi moduli sono ben distinti all'interno del nostro codice, cosa che rende le tre fasi facilmente modificabili. Data la struttura dell'implementazione anche la parte riguardante il multithread è facilmente modificabile per future revisioni.
Il progetto ha richiesto circa due mesi di sviluppo per essere completato. Si è svolto sotto la supervisione del prof. Vincenzo Deufemia e della dott.ssa Loredana Caruccio. Il lavoro si è svolto con i colleghi Ceruso Raffaele e Luigi Durso. Ognuno di noi si è occupato di una fase differente però si è resa necessaria una collaborazione per integrare e manutenere le fasi sviluppate. 